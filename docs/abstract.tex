\documentclass{sig-alternate-10pt}

\newcommand{\ttt}{\texttt}

\usepackage{graphicx}
\usepackage{changepage}
\usepackage{lipsum}

\title{FLASH\\Fast Linux Advanced Scheduler Hardware}
\author{
	Mark Aligbe\\
	    \email{ma2799@columbia.edu}
	\and
    Chae Jubb\\
        \email{ecj2122@columbia.edu}
}
\date{7 April 2015}

\begin{document}
\maketitle

\begin{abstract}
\lipsum[1]
\end{abstract}

\section{Introduction}
\lipsum[1]

\section{Scheduling in the Kernel}
\lipsum[1]

\section{FLASH Architecture}
The FLASH hardware unit is built with ease of replacement of the usual
software implementation in mind.  The major addition we add over a standard
scheduler is the ability for the scheduler itself to raise an interrupt to
indicate a tick. Figure~\ref{fig:arch_overview} gives an overview of the
system architecture.

\begin{figure}
	\begin{center}
		picture here.
		\caption{System Architecture}
		\label{fig:arch_overview}
	\end{center}
\end{figure}

We now describe in more detail both the interface and
internal implementation of the scheduling unit.

\subsection{FLASH Interface}
At its most basic, the hardware interface is given by two major segments:
\emph{scheduling control} and \emph{process control}.  Together these two
segments allow processes to be scheduled based on an up-to-date accounting
of process status, kept via the process control interface.  The scheduling
controller interface sends timer tick interrupts and serves incoming
requests for new processes to run.

\subsubsection{Process Control Interface}
Let us first focus deeply on this process control interface.  It is here
that FLASH is given information to hold the proper state of all running
processes.  Currently, we store process id (PID), priority, and state
triples as passed through the interface using a standard four-phase
handshake.

\paragraph{Consistency} A major hurdle with offloading scheduling decisions
to a dedicated piece of hardware is ensuring consistency between the data
structures maintained by the software kernel (again, here Linux) and the
hardware unit.

We cannot completely offload the data about processes to hardware most
obviously because of sheer size of objects like the \ttt{task\_struct} as
well as their high use by modules other than the scheduler.  We must,
though, retain some structure in FLASH memory so that it may make even the
most basic scheduling decisions.

For this reason, we expect each update in process priority or state to be
communicated to FLASH so it may continue making accurate scheduling
decisions.

\subsubsection{Scheduling Control Interface}

\section{Evaluation}
\lipsum[1]

\section{Integration}
\lipsum[1]

\section{Related Works}
\lipsum[1]

\section{Conclusion}
\lipsum[1]

\nocite{*}
{\small
	\bibliographystyle{abbrv}
	\bibliography{ref}
}

\end{document}
